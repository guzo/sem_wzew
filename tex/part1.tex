\section{Dostępne możliwości} %FIXME: mniej głupia nazwa
\subsection{Sprzęt}
%%%%%%%%%%%%%%%%%%%%%%%%%%%%%%%%%%%%%%%%%%%%%%%%%%%%%%%%%%%%%%%%%%%%%%%%%%%%%%%%
\begin{frame}[fragile]{x86\_64}
	\begin{block}{Nowości}
		\begin{itemize}
			\item Nowe rejestry: \textbf{SIMD} i \textbf{general purpose} (2x więcej niż w x86)
			\item Nowe instrukcje: \textbf{lea} - load effective address (z myślą o operacjach
			tablicowych, jednak można wykorzystać inaczej - przykład)
			\item Adresowanie 64-bitowe (większe zmienne + więcej pamięci)
			\item Rozszerzenia (dla i7): MMX, SSE, SSE2, SSE3, SSSE3, SSE4.1, SSE4.2, AVX, AES, PCLMUL;
		\end{itemize}
	\end{block}
\end{frame}
%%%%%%%%%%%%%%%%%%%%%%%%%%%%%%%%%%%%%%%%%%%%%%%%%%%%%%%%%%%%%%%%%%%%%%%%%%%%%%%%
\begin{frame}[fragile]{Kompilator}
	\begin{block}{Ogólnie}
	 Kompilator (z łaciny \textit{compilare}) - grabieżca %FIXME łacińskie literki w compilare
	\end{block}
	\begin{block}{GCC 1/2}
 		\begin{itemize}
			\item Optymalizacje ogólnie:
			\begin{itemize}
			 \item \verb*%-p[g]% //profilowanie
			 \item \verb*%-g[gdb],-Og% //debugowanie
			 \item \verb*%-O{0,1,2,3}%
			 \item \verb*%-ffast-math%
			\end{itemize}
 			\item Optymalizacje pod konkretną architekturę:
 			 \begin{itemize}
				\item \verb*%-m%
				\item \verb*%-mtune%
				\item \verb*%-march%
				\item \verb*%-mfpmath% (przykład)
 			 \end{itemize}
 			 \url{http://gcc.gnu.org/onlinedocs/gcc/i386-and-x86_002d64-Options.html}
 		\end{itemize}		
 	\end{block}
\end{frame}
%%%%%%%%%%%%%%%%%%%%%%%%%%%%%%%%%%%%%%%%%%%%%%%%%%%%%%%%%%%%%%%%%%%%%%%%%%%%%%%%
\begin{frame}[fragile]{Współpraca z kompilatorem}
	\begin{block}{Ogólnie}
	 Kompilator (z łaciny \textit{compilare}) - grabieżca %FIXME łacińskie literki w compilare
	\end{block}
	\begin{block}{GCC 2/2}
 		\begin{itemize}
			\item Warningi:
			\begin{itemize}
			 \item \verb*%-Wall%
			 \item \verb*%-Wextra%
			 \item \verb*%-Werror%
			 \item \verb*%-pedantic%
			 \item \verb*%-Wfloat-equal% //ciekawostka
			\end{itemize}
 			\item Informacja o użytych optymalizacjach:
 			 \begin{itemize}
				\item \verb*%-fstack-usage%
				\item \verb*%-ftree-vectorizer-verbose%
				\item \verb*%-fdump-tree-{vectorize,optimize}=stderr%
				\item \verb*%-fopt-info-{optimized,vec-missed}%
 			 \end{itemize}
 		\end{itemize}		
 	\end{block}
\end{frame}
%%%%%%%%%%%%%%%%%%%%%%%%%%%%%%%%%%%%%%%%%%%%%%%%%%%%%%%%%%%%%%%%%%%%%%%%%%%%%%%%
\begin{frame}[fragile]{Współpraca z kompilatorem}
	\begin{block}{Builtins/Intrinsics}
		Funkcje i typy wbudowane w kompilator na przykład:
		\begin{itemize}
			\item \verb*%__builtin_expects({G_LIKELY,G_UNLIKELY})%
			\item \verb*%__builtin_cpu_supports("sse2")%
			\item \verb*%__m64 avariable%
		\end{itemize}
		\url{http://gcc.gnu.org/onlinedocs/gcc/X86-Built_002din-Functions.html}
	\end{block}
	\begin{block}{Wektoryzcja pętli (builtin)}
 		\begin{itemize}
			\item \verb*%__m64 _mm_add_pi16%
			\item \verb*%__m64 _mm_mullo_pi16%
			\item \verb*%__m64 _mm_min_ps%
 		\end{itemize}
 		\url{http://gcc.gnu.org/onlinedocs/gcc-4.8.0/gcc/Vector-Extensions.html}
 	\end{block}
\end{frame}
%%%%%%%%%%%%%%%%%%%%%%%%%%%%%%%%%%%%%%%%%%%%%%%%%%%%%%%%%%%%%%%%%%%%%%%%%%%%%%%%
\begin{frame}[fragile]{Współpraca z kompilatorem}
	Language extensions? Nie widze sensu, bo o tym już było. Pogadamy.
\end{frame}
%%%%%%%%%%%%%%%%%%%%%%%%%%%%%%%%%%%%%%%%%%%%%%%%%%%%%%%%%%%%%%%%%%%%%%%%%%%%%%%%
%%%%%%%%%%%%%%%%%%%%%%%%%%%%%%%%%%%%%%%%%%%%%%%%%%%%%%%%%%%%%%%%%%%%%%%%%%%%%%%%
\begin{frame}[fragile]{Współpraca z kompilatorem}
	\begin{block}{Attributes}
		Pozwala nadawać atrybuty specjalne zmiennym, strukturom danych, typom, funkcjom.
		\begin{itemize}
			\item \verb*%int x __attribute__((aligned(16)))=0;% //zmienna
			\item \verb*%int x[2] __attribute__ ((packed));% //jako czesc struktury %FIXME kod ze strukturą
			\item \verb*%typedef int more_aligned_int %
			\verb*%__attribute__ ((aligned (8)));% //typ %FIXME Monkey patch
			\item \verb*%int old_fn () __attribute__ ((fastcall));% //funkcja
			\item \verb*%void fatal () __attribute__ ((noreturn));%
		\end{itemize}
		\url{http://gcc.gnu.org/onlinedocs/gcc/Function-Attributes.html}
	\end{block}
\end{frame}
%\subsection{Język}
%\subsection{Biblioteki}
%\subsection{Narzędzia}