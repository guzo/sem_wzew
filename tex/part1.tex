\section{Dostępne możliwości} %FIXME: mniej głupia nazwa
\subsection{Sprzęt}
%%%%%%%%%%%%%%%%%%%%%%%%%%%%%%%%%%%%%%%%%%%%%%%%%%%%%%%%%%%%%%%%%%%%%%%%%%%%%%%%
\begin{frame}[fragile]{x86\_64}
	\begin{block}{Nowości}
		\begin{itemize}
			\item Nowe rejestry: \textbf{SIMD} i \textbf{general purpose} (2x więcej niż w x86)
			\item Nowe instrukcje: \textbf{lea} - load effective address (z myślą o operacjach
			tablicowych, jednak można wykorzystać inaczej - przykład)
			\item Adresowanie 64-bitowe (większe zmienne + więcej pamięci)
			\item Rozszerzenia (dla i7): MMX, SSE, SSE2, SSE3, SSSE3, SSE4.1, SSE4.2, AVX, AES, PCLMUL;
		\end{itemize}
	\end{block}
\end{frame}
%%%%%%%%%%%%%%%%%%%%%%%%%%%%%%%%%%%%%%%%%%%%%%%%%%%%%%%%%%%%%%%%%%%%%%%%%%%%%%%%
\begin{frame}[fragile]{Kompilator}
	\begin{block}{Ogólnie}
	 Kompilator (z łaciny \textit{compilare}) - grabieżca %FIXME łacińskie literki w compilare
	\end{block}
	\begin{block}{GCC 1/3}
 		\begin{itemize}
			\item Optymalizacje ogólnie:
			\begin{itemize}
			 \item \verb*%-p[g]% //profilowanie
			 \item \verb*%-g[gdb],-Og% //debugowanie
			 \item \verb*%-O{0,1,2,3}%
			 \item \verb*%-ffast-math%
			\end{itemize}
 			\item Optymalizacje pod konkretną architekturę:
 			 \begin{itemize}
				\item \verb*%-m%
				\item \verb*%-mtune%
				\item \verb*%-march%
				\item \verb*%-mfpmath% (przykład)
 			 \end{itemize}
 			 \url{http://gcc.gnu.org/onlinedocs/gcc/i386-and-x86_002d64-Options.html}
 		\end{itemize}		
 	\end{block}
\end{frame}
%%%%%%%%%%%%%%%%%%%%%%%%%%%%%%%%%%%%%%%%%%%%%%%%%%%%%%%%%%%%%%%%%%%%%%%%%%%%%%%%
%\subsection{Język}
%\subsection{Biblioteki}
%\subsection{Narzędzia}